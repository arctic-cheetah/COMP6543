\documentclass{article}
\usepackage{amsmath} 
\usepackage{algorithm}
\usepackage{algorithmic}
\title{COMP6453 Assignment I: Q4}
\author{Joules Patman}
\begin{document}
\maketitle


Q4)

a)
The maximum number of calls to the SHA-1 function is if we take the first 40 bits (5 bytes) 
and try to find a collsion for $H(m_1) = H(m_k)$. This is the worst case 
scenario and $2^{40}$ calls will be required
\\
\\
b)
But By using the birthday attack we can use the fact that the actual bits of security will be $2^{n/2}$
where is the number of bits, until we find a collision with probability of 50\%. Thus we will need to 
compare $2^{20}$ bits instead.
\\
\\

c) Pseudocode:
\begin{algorithm}
\caption{Find Colliding Hash}
\begin{algorithmic}[1]
\STATE \textbf{Define function:}
\STATE \hspace{1em} \textbf{Define} \texttt{HashTable} \COMMENT{hash table stores (hash, message) pairs}
\STATE \hspace{1em} $(m_1, h_1) \leftarrow$ \texttt{get40BitsFromSha1()}
\STATE \hspace{1em} \texttt{didFindHash} $\leftarrow$ \texttt{HashTable.get}($h_1$)
\IF{\texttt{didFindHash} $\neq$ \texttt{null}}
    \STATE \hspace{2em} \texttt{HashTable.add}($h_1$)
\ELSIF{$m_1 == \texttt{val}$}
    \STATE \hspace{2em} \texttt{skip}
\ELSE
    \STATE \hspace{2em} $m_2 \leftarrow$ \texttt{didFindHash}
\ENDIF
\STATE \textbf{return} ($h_1, m_1, m_2$)
\end{algorithmic}
\end{algorithm}


% \begin{lstlisting}[language = <language>, other options]
% \end{listing}
% define function:   
%     // Hash table defines (hash, message) pair
%     define HashTable

%     (m1, h1) = get40BitsFromSha1()

%     didFindHash = HashTable.get(h1)    

%     if (didFindHash != null) {
%         HashTable.add(h1)
%     }
%     else if (m1 == val) {
%         skip
%     }
%     else {
%         m2 = didFindHash
%     }

%     return (h1, m1, m2)


\end{document}