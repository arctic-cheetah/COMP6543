\documentclass[10pt]{article}
\usepackage{amsmath} 
\title{COMP6453 Assignment I}
\begin{document}
\maketitle

Recall the definition of perfect secrecy, is that the presence of the ciphertext $c$
does not reveal information about the plain text $m$ meaning that
\begin{subequations}
    \begin{align}
        P(M = m | C = c) = P(M = m) 
    \end{align}
\end{subequations}
and $m \in M$ $c \in C$ and $k \in K$. Where $M,K,C$ are their respective spaces.

This also implies by Bayes Theorem that:
\begin{subequations}
    \begin{align}
        P(C = c | M = m) * P(M = m) = P(M = m | C = c) * P(C = c)
    \end{align}
\end{subequations}
Because $P(M = m) > 0$ and $P(C = c) > 0$ and this implies:
$P(C = c|M = m) = P(C = c) (2\text{b})$.



For some encryption scheme with $E$, for it
to be considered perfectly secret,
it must be shown that it is uniformly distributed 
over M,C with Key Space K, for any message $m \in M$ and 
any cipher text $c \in C$;

Then every $P(M = m) > 0$ and $P(C = c) > 0$ and that
$E(k, m_0) = c_0$ and $E(k, m_1) = c_0$ for $k \in K$.
\begin{subequations}
    \begin{align}
    P(C = c | M = m) &= P(E(k, M) = c| M = m) 
    \\
    &= P(E(k, m) = c | M = m)
    \\
    &= P(E(k, m) = c) 
    \end{align}
\end{subequations}
3a) is by defintion the encryption of $m$ that is $E(k, m) = c$.
3b) is because we condition on the event that some plaintext $M = m_0$ 
Then it 3c) occurs because the key $k$ is independent from the message space where 
$M = m$

Therfore, using equation 2b), 3c) and $m_0, m_1 \in M$:
\begin{subequations}
    \begin{align}
        P(E(k, m_0) = c) &= P(C = c | M = m_0)
        \\
        &= P(C = c)
        \\
        &= P(C = c | M = m_1) = P(E(k, m_0) = c)
    \end{align}
\end{subequations}

So when $P(E(k, m_0) = c)$ and $P(E(k, m_1) = c)$ for some $c$. This means when 
$E(k, m_1) = c_1$ and $E(k, m_0) = c_0$ then $P(E(k, m_0) = c_0) = P(E(k, m_1) = c_1)$.

Therefore $P(C = c_0) = P(C = c_1)$.
Thus the encryption mechanism provided has perfect secrecy.
QED.

% Ergo, by induction we see for any $c_n \in C$, each of them will be occur equally likely
% ie: $P(E(k, m_0) = c_0) = P(E(k, m_1) = c_1) = P(E(k, m_2) = c_2)... = P(E(k, m_n) = c_n)$
% and in turn uniformly distributed.
% And the encryption mechanism has perfect secrecy


\end{document}